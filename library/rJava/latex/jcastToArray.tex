\HeaderA{jcastToArray}{Ensures that a given object is an array reference}{jcastToArray}
\aliasA{.jcastToArray}{jcastToArray}{.jcastToArray}
\keyword{interface}{jcastToArray}
\begin{Description}\relax
\code{.jcastToArray} takes a Java object reference of any kind and
returns Java array reference if the given object is a reference to an
array.
\end{Description}
\begin{Usage}
\begin{verbatim}
.jcastToArray(obj, signature=NULL, class="", quiet=FALSE)
\end{verbatim}
\end{Usage}
\begin{Arguments}
\begin{ldescription}
\item[\code{obj}] Java object reference to cast or a scalar vector
\item[\code{signature}] array signature in JNI notation (e.g. \code{"[I"} for
an array of integers). If set to \code{NULL} (the default),
the signature is automatically determined from the object's class.
\item[\code{class}] force the result to pose as a particular Java
class. This has the same effect as using \code{\LinkA{.jcast}{.jcast}} on the
result and is provided for convenience only.
\item[\code{quiet}] if set to \code{TRUE}, no failures are reported and the
original object is returned unmodified.
\end{ldescription}
\end{Arguments}
\begin{Details}\relax
Sometimes a result of a method is by definition of the class
\code{java.lang.Object}, but the acutal referenced object may be an
array. In that case the method returns a Java object reference instead
of an array reference. In order to obtain an array reference, it is
necessary to cast such an object to an array reference - this is done
using the above \code{.jcastToArray} function.

The input is an object reference that points to an array. Ususally the
signature should be left at \code{NULL} such that it is determined
from the object's class. This is also a check, because if the object's
class is not an array, then the functions fails either with an error
(when \code{quiet=FALSE}) or by returing the original object (when
\code{quiet=TRUE}). If the signature is set to anything else, it is
not verified and the array reference is always created, even if it may
be invalid and unusable.

For convenience \code{.jcastToArray} also accepts non-references in
which case it simply calls \code{\LinkA{.jarray}{.jarray}}, ignoring all other
parameters.
\end{Details}
\begin{Value}
Returns a Java array reference (\code{jarrayRef}) on success. If
\code{quiet} is \code{TRUE} then the result can also be the original
object in the case of failure.
\end{Value}
\begin{Examples}
\begin{ExampleCode}
## Not run: 
a <- .jarray(1:10)
print(a)
# let's create an array containing the array
aa <- .jarray(list(a))
print(aa)
ba <- .jevalArray(aa)[[1]]
# it is NOT the inverse, because .jarray works on a list of objects
print(ba)
# so we need to cast the object into an array
b <- .jcastToArray(ba)
# only now a and b are the same array reference
print(b)
# for convenience .jcastToArray behaves like .jarray for non-references
print(.jcastToArray(1:10/2))
## End(Not run)
\end{ExampleCode}
\end{Examples}

