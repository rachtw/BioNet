\HeaderA{jarray}{Java array handling functions}{jarray}
\aliasA{.jarray}{jarray}{.jarray}
\aliasA{.jevalArray}{jarray}{.jevalArray}
\keyword{interface}{jarray}
\begin{Description}\relax
\code{.jarray} takes a vector (or a list of Java references) as its
argument, creates a Java array containing the elements of the vector
(or list) and returns a reference to such newly created array.

\code{.jevalArray} takes a reference to a Java array and returns its
contents (if possible).
\end{Description}
\begin{Usage}
\begin{verbatim}
.jarray(x, contents.class = NULL)
.jevalArray(obj, rawJNIRefSignature = NULL, silent = FALSE)
\end{verbatim}
\end{Usage}
\begin{Arguments}
\begin{ldescription}
\item[\code{x}] vector or a list of Java references
\item[\code{contents.class}] common class of the contained objects, see
details
\item[\code{obj}] Java object reference to an array that is to be evaluated
\item[\code{rawJNIRefSignature}] JNI signature that whould be used for
conversion. If set to \code{NULL}, the signature is detected
automatically.
\item[\code{silent}] if set to true, warnings are suppressed
\end{ldescription}
\end{Arguments}
\begin{Details}\relax
\code{.jarray}: The input can be either a vector of some sort (such as
numeric, integer, logical, ...) or a list of Java references. The
contents is pushed to the Java side and a corresponding array is
created. The type of the array depends on the input vector type. For
example numeric vector creates \code{double[]} array, integer vector
creates \code{int[]} array, character vector \code{String[]} array and
so on. If \code{x} is a list, it must contain Java references only (or
\code{NULL}s which will be treaded as \code{NULL} references).

The \code{contents.class} parameter is used only if \code{x} is a list
of Java object references and it can specify the class that will be
used for all objects in the array. If set to \code{NULL} no assumption
is made and \code{java/lang/Object} will be used. Use with care and
only if you know what you're doing - you can always use
\code{\LinkA{.jcast}{.jcast}} to cast the entire array to another type even if
you use a more general object type.

The result is a reference to the newly created array.

The inverse function which fetches the elements of an array reference
is \code{.jevalArray}.

\code{.jevalArray} currently supports only a subset of all possible
array types. Recursive arrays are handled by returning a list of
references which can then be evaluated separately.
\end{Details}
\begin{Value}
\code{.jarray} returns a Java array reference (\code{jarrayRef}) to an
array created with the supplied contents.

\code{.jevalArray} returns the contents of the array object.
\end{Value}
\begin{Examples}
\begin{ExampleCode}
## Not run: 
a <- .jarray(1:10)
print(a)
.jevalArray(a)
b <- .jarray(c("hello","world"))
print(b)
c <- .jarray(list(a,b))
print(c)
## End(Not run)
\end{ExampleCode}
\end{Examples}

