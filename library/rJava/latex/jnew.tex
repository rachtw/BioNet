\HeaderA{jnew}{Create a Java object}{jnew}
\aliasA{.jnew}{jnew}{.jnew}
\keyword{interface}{jnew}
\begin{Description}\relax
\code{.jnew} create a new Java object
\end{Description}
\begin{Usage}
\begin{verbatim}
.jnew(class, ..., check=TRUE, silent=!check)
\end{verbatim}
\end{Usage}
\begin{Arguments}
\begin{ldescription}
\item[\code{class}] fully qualified class name in JNI notation (e.g. \code{"java/lang/String"}).
\item[\code{...}] Any parametes that will be passed to the corresponding
constructor. The parameter types are determined automatically and/or
taken from the \code{jobjRef} object. For details see
\code{\LinkA{.jcall}{.jcall}}. Note that all named parameters are discarded.
\item[\code{check}] If set to \code{TRUE} then \code{\LinkA{.jcheck}{.jcheck}} is invoked before
and after the call to the constructor to clear any pending Java
exceptions.
\item[\code{silent}] If set to \code{FALSE} then \code{.jnew} will fail with an error if
the object cannot be created, otherwise a null-reference is returned
instead. In addition, this flag is also passed to final
\code{.jcheck} if \code{check} above is set to \code{TRUE}. Note
that the error handling also clears exceptions, so
\code{check=FALSE, silent=FALSE} is usually not a meaningful
combination.

\end{ldescription}
\end{Arguments}
\begin{Value}
Returns the reference (\code{jobjRef}) to the newly created object or
\code{null}-reference (see \code{\LinkA{.jnull}{.jnull}}) if something went wrong.
\end{Value}
\begin{SeeAlso}\relax
\code{\LinkA{.jcall}{.jcall}}, \code{\LinkA{.jnull}{.jnull}}
\end{SeeAlso}
\begin{Examples}
\begin{ExampleCode}
## Not run: 
f <- .jnew("java/awt/Frame","Hello")
.jcall(f,,"setVisible",TRUE)
## End(Not run)
\end{ExampleCode}
\end{Examples}

