\HeaderA{jreflection}{Simple helper functions for Java reflection}{jreflection}
\aliasA{.jconstructors}{jreflection}{.jconstructors}
\aliasA{.jfields}{jreflection}{.jfields}
\aliasA{.jmethods}{jreflection}{.jmethods}
\keyword{interface}{jreflection}
\begin{Description}\relax
\code{.jconstructors} returns a character vector with all constructors for
a given class or object.
\code{.jmethods} returns a character vector with all methods for
a given class or object.
\code{.jfields} returns a character vector with all fileds (aka attributes) for a given class or object.
\end{Description}
\begin{Usage}
\begin{verbatim}
.jconstructors(o)
.jmethods(o, name = NULL)
.jfields(o)
\end{verbatim}
\end{Usage}
\begin{Arguments}
\begin{ldescription}
\item[\code{o}] Name of a class (either notation is fine) or an object whose
class will be queried
\item[\code{name}] Name of the method to look for. May contain regular
expressions except for \code{\textasciicircum{}\$}.
\end{ldescription}
\end{Arguments}
\begin{Details}\relax
There first two functions are intended to help with finding correct
signatures for methods and constructors. Since the low-level API in rJava doesn't use reflection automatically, it is necessary to provide a proper
signature. That is somewhat easier using the above methods.
\end{Details}
\begin{Value}
Returns a character vector. Each entry corresponds to
\code{toString()} call on the \code{Constructor} resp. \code{Method} resp. \code{Field}
object.
\end{Value}
\begin{SeeAlso}\relax
\code{\LinkA{.jcall}{.jcall}}, \code{\LinkA{.jnew}{.jnew}}, \code{\LinkA{.jcast}{.jcast}} or \code{\LinkA{\$,jobjRef-method}{.Rdol.,jobjRef.Rdash.method}}
\end{SeeAlso}
\begin{Examples}
\begin{ExampleCode}
## Not run: 
.jconstructors("java/util/Vector")
v <- .jnew("java/util/Vector")
.jmethods(v, "add")
## End(Not run)
\end{ExampleCode}
\end{Examples}

