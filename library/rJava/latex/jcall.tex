\HeaderA{jcall}{Call a Java method}{jcall}
\aliasA{.jcall}{jcall}{.jcall}
\keyword{interface}{jcall}
\begin{Description}\relax
\code{.jcall} calls a Java method with the supplied arguments.
\end{Description}
\begin{Usage}
\begin{verbatim}
.jcall(obj, returnSig = "V", method, ..., evalArray = TRUE, 
    evalString = TRUE, check = TRUE, interface = "RcallMethod")
\end{verbatim}
\end{Usage}
\begin{Arguments}
\begin{ldescription}
\item[\code{obj}] Java object (\code{jobjRef} as returned by
\code{\LinkA{.jcall}{.jcall}} or \code{\LinkA{.jnew}{.jnew}}) or fully qualified
class name in JNI notation (e.g. \code{"java/lang/String"}).
\item[\code{returnSig}] Return signature in JNI notation (e.g. "V" for void,
"[I" for \code{int[]} etc.). For convenience additional type
\code{"S"} is supported and expanded to
\code{"Ljava/lang/String;"}, re-mapping \code{"T"} to represent the
type \code{short}.
\item[\code{method}] The name of the method to be called
\item[\code{...}] Any parametes that will be passed to the Java method. The parameter
types are determined automatically and/or taken from the
\code{jobjRef} object. All named parameters are discarded.
\item[\code{evalArray}] This flag determines whether the array return value
is evaluated (\code{TRUE}) or passed back as Java object reference
(\code{FALSE}).
\item[\code{evalString}] This flag determines whether string result is returned
as characters or as Java object reference.
\item[\code{check}] If set to \code{TRUE} then checks for exceptions are
performed before and after the call using
\code{\LinkA{.jcheck}{.jcheck}(silent=FALSE)}. This is usually the desired
behavior, because all calls fail until an expection is cleared.
\item[\code{interface}] This option is experimental and specifies the
interface used for calling the Java method; the current
implementation supports two interfaces:
\Itemize{
\item[\code{"RcallMethod"}] the default interface.
\item[\code{"RcallSyncMethod"}] synchronized call of a
method. This has simmilar effect as using \code{synchronize} in
Java.
}

\end{ldescription}
\end{Arguments}
\begin{Details}\relax
\code{.jcall} requires exact match of argument and return types. For
higher efficiency \code{.jcall} doens't perform any lookup in the
reflection tables. This means that passing subclasses of the classes
present in the method definition requires explicit casting using
\code{\LinkA{.jcast}{.jcast}}. Passing \code{null} arguments also needs a
proper class specification with \code{\LinkA{.jnull}{.jnull}}.

Java types \code{long} and \code{float} have no corresponding types in
R and therefore any such parameters must be flagged as such using
\code{\LinkA{.jfloat}{.jfloat}} and \code{\LinkA{.jlong}{.jlong}} functions respectively.

Java also distinguishes scalar and array types whereas R doesn't have
the concept of a scalar. In R a scalar is basically a vector (called
array in Java-speak) of the length 1. Therefore passing vectors of the
length 1 is ambiguous. \code{.jcall} assumes that any vector of the
length 1 that corresponds to a native Java type is a scalar. All other
vectors are passed as arrays. Therefore it is important to use
\code{\LinkA{.jarray}{.jarray}} if an arbitrary vector (including those of the
length 1) is to be passed as an array parameter.
\end{Details}
\begin{Value}
Returns the result of the method.
\end{Value}
\begin{SeeAlso}\relax
\code{\LinkA{.jnew}{.jnew}}, \code{\LinkA{.jcast}{.jcast}}, \code{\LinkA{.jnull}{.jnull}},
\code{\LinkA{.jarray}{.jarray}}
\end{SeeAlso}
\begin{Examples}
\begin{ExampleCode}
## Not run: 
.jcall("java/lang/System","S","getProperty","os.name")
f <- .jnew("java/awt/Frame","Hello")
.jcall(f,,"setVisible",TRUE)
## End(Not run)
\end{ExampleCode}
\end{Examples}

