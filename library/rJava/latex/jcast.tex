\HeaderA{jcast}{Cast a Java object to another class}{jcast}
\aliasA{.jcast}{jcast}{.jcast}
\keyword{interface}{jcast}
\begin{Description}\relax
\code{.jcast} returns a Java object reference cast to another Java class.
\end{Description}
\begin{Usage}
\begin{verbatim}
.jcast(obj, new.class = "java/lang/Object")
\end{verbatim}
\end{Usage}
\begin{Arguments}
\begin{ldescription}
\item[\code{obj}] a Java object reference
\item[\code{new.class}] fully qualified class name in JNI notation
(e.g. \code{"java/lang/String"}). Although rJava itself performs no
type check in \code{.jcast}, Java will produce an exception on the
first use if the cast is illegal.
\end{ldescription}
\end{Arguments}
\begin{Details}\relax
This function is necessary if a argument of \code{\LinkA{.jcall}{.jcall}} or
\code{\LinkA{.jnew}{.jnew}} is defined as the superclass of the object to be
passed. (See \code{\LinkA{.jcall}{.jcall}}) No type check is performed and the
original object is not modified.
\end{Details}
\begin{Value}
Returns a Java object reference (\code{jobjRef}) to the object
\code{obj}, changing the object class.
\end{Value}
\begin{SeeAlso}\relax
\code{\LinkA{.jcall}{.jcall}}
\end{SeeAlso}
\begin{Examples}
\begin{ExampleCode}
## Not run: 
v <- .jnew("java/util/Vector")
.jcall("java/lang/System","I","identityHashCode",.jcast(v, "java/lang/Object"))
## End(Not run)
\end{ExampleCode}
\end{Examples}

