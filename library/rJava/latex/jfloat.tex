\HeaderA{jfloat}{Wrap numeric vector as flat Java parameter}{jfloat}
\aliasA{.jbyte}{jfloat}{.jbyte}
\aliasA{.jchar}{jfloat}{.jchar}
\aliasA{.jfloat}{jfloat}{.jfloat}
\aliasA{.jlong}{jfloat}{.jlong}
\aliasA{jbyte}{jfloat}{jbyte}
\aliasA{jchar}{jfloat}{jchar}
\aliasA{jlong}{jfloat}{jlong}
\keyword{interface}{jfloat}
\begin{Description}\relax
\code{.jfloat} marks a numberic vector as an object that can be used
as parameter to Java calls that require \code{float} parameters.
Similarly, \code{.jlong} marks a numeric vector as \code{long} parameter.
\end{Description}
\begin{Usage}
\begin{verbatim}
.jfloat(x)
.jlong(x)
.jbyte(x)
.jchar(x)
\end{verbatim}
\end{Usage}
\begin{Arguments}
\begin{ldescription}
\item[\code{x}] numeric vector
\end{ldescription}
\end{Arguments}
\begin{Details}\relax
R has no native \code{float} or \code{long} type. Numeric vectors are
stored as \code{double}s, hence there is no native way to pass float
numbers to Java methods. The \code{.jfloat} call marks a numeric
vector as having the Java type \code{float} by wrapping it in the
\code{jfloat} class. The class is still a subclass of \code{numeric},
therefore all regular R operations are unaffected by this.

Similarly, \code{.jlong} is used to mark a numeric vector as a
parameter of the \code{long} Java type. Please note that in general R
has no native type that will hold a \code{long} value, so conversion
between Java's \code{long} type and R's numeric is potentially lossy.

\code{.jbyte} is used when a scalar byte is to be passed ot Java. Note
that byte arrays are natively passed as RAW vectors, not as
\code{.jbyte} arrays.

\code{jchar} is strictly experimental and may be based on
\code{character} vectors in the future.
\end{Details}
\begin{Value}
Returns a numeric vector of the class \code{jfloat}, \code{jlong},
\code{jbyte} or \code{jchar}
that can be used as parameter to Java calls that require
\code{float}, \code{long}, \code{byte} or \code{char} parameters
respectively.
\end{Value}
\begin{SeeAlso}\relax
\code{\LinkA{.jcall}{.jcall}}, \code{\LinkA{jfloat-class}{jfloat.Rdash.class}}
\end{SeeAlso}

