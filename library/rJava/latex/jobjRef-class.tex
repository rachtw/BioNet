\HeaderA{jobjRef-class}{Class "jobjRef" - Reference to a Java object}{jobjRef.Rdash.class}
\keyword{classes}{jobjRef-class}
\begin{Description}\relax
This class describes a reference to an object held in a JavaVM.
\end{Description}
\begin{Section}{Objects from the Class}
Objects of this class should *not* be created directly. Instead, the function \code{\LinkA{.jnew}{.jnew}} should be use to create new Java objects. They can also be created as results of the \code{\LinkA{.jcall}{.jcall}} function.
\end{Section}
\begin{Section}{Slots}
\describe{
\item[\code{jobj}:] Internal identifier of the object (external pointer to be precise)
\item[\code{jclass}:] Java class name of the object (in JNI notation)
}
Java-side attributes are not accessed via slots, but the \code{\$} operator instead.
\end{Section}
\begin{Section}{Methods}
This object's Java methods are not accessed directly. Instead, \code{\LinkA{.jcall}{.jcall}} JNI-API should be used for invoking Java methods. For convenience the \code{\$} operator can be used to call methods via reflection API.
\end{Section}
\begin{Author}\relax
Simon Urbanek
\end{Author}
\begin{SeeAlso}\relax
\code{\LinkA{.jnew}{.jnew}}, \code{\LinkA{.jcall}{.jcall}}  or \code{\LinkA{jarrayRef-class}{jarrayRef.Rdash.class}}
\end{SeeAlso}

