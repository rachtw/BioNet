\HeaderA{JavaAccess}{Field/method operator for Java objects}{JavaAccess}
\aliasA{\$,jobjRef-method}{JavaAccess}{.Rdol.,jobjRef.Rdash.method}
\aliasA{\$<-,jobjRef-method}{JavaAccess}{.Rdol.<.Rdash.,jobjRef.Rdash.method}
\keyword{interface}{JavaAccess}
\begin{Description}\relax
The \code{\$} operator for \code{jobjRef} Java object references provides convenience access to object attributes and calling Java methods.
\end{Description}
\begin{Details}\relax
rJava provies two levels of API: low-level JNI-API in the form of \code{\LinkA{.jcall}{.jcall}} function and high-level reflection API based on the \code{\$} operator. The former is very fast, but inflexible. The latter is a convenient way to use Java-like programming at the cost of performance. The reflection API is build around the \code{\$} operator on \code{\LinkA{jobjRef-class}{jobjRef.Rdash.class}} objects that allows to access Java attributes and call object methods.

\code{\$} returns either the value of the attribute or calls a method, depending on which name matches first.

\code{\$<-} assigns a value to the corresponding Java attribute.

This is just a convenience API. Internally all calls are mapped into \code{\LinkA{.jcall}{.jcall}} calls, therefore the calling conventions and returning objects use the smae rules. For time-critical Java calls \code{\LinkA{.jcall}{.jcall}} should be used directly.

NOTE: This interface is still very experimental! Some type conversions may not work or it may fail to find the method you want. You may need to use \code{\LinkA{.jcall}{.jcall}}, consult \code{\LinkA{.jmethods}{.jmethods}}, \code{\LinkA{.jfields}{.jfields}} and possibly apply \code{\LinkA{.jcast}{.jcast}} in such cases. If it is a general problem, please report such issues whith a small reproducible example.
\end{Details}
\begin{Section}{Methods}
\describe{
\item[\code{\$}] \code{signature(x = "jobjRef")}: ... 
\item[\code{\$<-}] \code{signature(x = "jobjRef")}: ... 
}
\end{Section}
\begin{SeeAlso}\relax
\code{\LinkA{.jcall}{.jcall}}, \code{\LinkA{.jnew}{.jnew}}, \code{\LinkA{jobjRef-class}{jobjRef.Rdash.class}}
\end{SeeAlso}
\begin{Examples}
\begin{ExampleCode}
## Not run: 
v <- .jnew("java/lang/String","Hello World!")
v$length()
v$indexOf("World")
## End(Not run)
\end{ExampleCode}
\end{Examples}

