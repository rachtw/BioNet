\HeaderA{jequals}{Comparing Java References}{jequals}
\aliasA{"!=,ANY,jobjRef-method}{jequals}{"!=,ANY,jobjRef.Rdash.method}
\aliasA{"!=,jobjRef,ANY-method}{jequals}{"!=,jobjRef,ANY.Rdash.method}
\aliasA{"!=,jobjRef,jobjRef-method}{jequals}{"!=,jobjRef,jobjRef.Rdash.method}
\aliasA{.jequals}{jequals}{.jequals}
\aliasA{==,ANY,jobjRef-method}{jequals}{==,ANY,jobjRef.Rdash.method}
\aliasA{==,jobjRef,ANY-method}{jequals}{==,jobjRef,ANY.Rdash.method}
\aliasA{==,jobjRef,jobjRef-method}{jequals}{==,jobjRef,jobjRef.Rdash.method}
\keyword{interface}{jequals}
\begin{Description}\relax
\code{.jequals} function can be used to determine whether two objects
are equal. In addition, it allows mixed comparison of non-Java object
for convenience, unless strict comparison is desired.

The binary operators \code{==} and \code{!=} are mapped to
(non-strict) call to \code{.jequals} for convenience.
\end{Description}
\begin{Usage}
\begin{verbatim}
.jequals(a, b, strict = FALSE)
\end{verbatim}
\end{Usage}
\begin{Arguments}
\begin{ldescription}
\item[\code{a}] first object
\item[\code{b}] second object
\item[\code{strict}] when set to \code{TRUE} then non-references save for
\code{NULL} are always treated as different, see details.
\end{ldescription}
\end{Arguments}
\begin{Details}\relax
Comparing two Java objects is performed by calling \code{equals}
method of one of the objects and passing the other object as its
argument. This allows Java objects to define the `equality' in
object-dependent way.

In addition, \code{.jequals} allows the comparison of Java object to
other scalar R objects. This is done by creating a temporary Java
object that corresponds to the R object and using it for a call to the
\code{equals} method. If such conversion is not possible a warning is
produced and the result it \code{FALSE}. The automatic conversion
will be avoided if \code{strict} parameter is set to \code{TRUE}.

\code{NULL} values in \code{a} or \code{b} are replaced by Java
\code{null}-references and thus \code{.jequals(NULL,NULL)} is \code{TRUE}.

If neither \code{a} and \code{b} are Java objects (with the exception
of both being \code{NULL}) then the result is identical to that of
\code{all.equal(a,b)}.

Neither comparison operators nor \code{.jequals} supports vectors and
returns \code{FALSE} in that case. A warning is also issued unless
strict comparison was requested.
\end{Details}
\begin{Value}
\code{TRUE} if both object are considered equal, \code{FALSE} otherwise.
\end{Value}
\begin{Section}{Methods}
\describe{
\item[!=] \code{signature(e1 = "ANY", e2 = "jobjRef")}: ... 
\item[!=] \code{signature(e1 = "jobjRef", e2 = "jobjRef")}: ... 
\item[!=] \code{signature(e1 = "jobjRef", e2 = "ANY")}: ... 
\item[==] \code{signature(e1 = "ANY", e2 = "jobjRef")}: ... 
\item[==] \code{signature(e1 = "jobjRef", e2 = "jobjRef")}: ... 
\item[==] \code{signature(e1 = "jobjRef", e2 = "ANY")}: ... 
}
\end{Section}
\begin{Note}\relax
Don't use \code{x == NULL} to check for
\code{null}-references, because \code{x} could be \code{NULL} and thus
the result would be an empty vector. Use \code{\LinkA{is.jnull}{is.jnull}}
instead.
(In theory \code{is.jnull} and \code{x == .jnull()} are the the same,
but \code{is.jnull} is more efficient.)
\end{Note}
\begin{SeeAlso}\relax
\code{\LinkA{is.jnull}{is.jnull}}
\end{SeeAlso}
\begin{Examples}
\begin{ExampleCode}
## Not run: 
s <- .jnew("java/lang/String", "foo")
.jequals(s, "foo") # TRUE
.jequals(s, "foo", strict=TRUE) # FALSE - "foo" is not a Java object
t <- s
.jequals(s, t, strict=TRUE) # TRUE

s=="foo" # TRUE
## End(Not run)
\end{ExampleCode}
\end{Examples}

