\HeaderA{jfiled}{Obtains the value of a field}{jfiled}
\aliasA{.jfield}{jfiled}{.jfield}
\keyword{interface}{jfiled}
\begin{Description}\relax
\code{.jfield} returns the value of the specified field on an object.
\end{Description}
\begin{Usage}
\begin{verbatim}
.jfield(o, name, simplify=TRUE, true.class=TRUE)
\end{verbatim}
\end{Usage}
\begin{Arguments}
\begin{ldescription}
\item[\code{o}] Class name or object (Java reference) whose field is to be
accessed. Static fields are supported both by specifying the class
name or using an instance.
\item[\code{name}] name of the field to access
\item[\code{simplify}] field accessor methods always return Java objects. If
set to \code{TRUE}, such object will be converted to scalar type if
possible (see \code{\LinkA{.jsimplify}{.jsimplify}}).
\item[\code{true.class}] field accessor methods always return objects of the
class \code{java.lang.Object}. If set to \code{TRUE}, the true Java
class is determined and the returned object's class will be
adjusted.
\end{ldescription}
\end{Arguments}
\begin{Details}\relax
\code{.jfield} uses reflection to access value of a field of a given
object.
\end{Details}
\begin{Value}
Contents of the field.
\end{Value}
\begin{SeeAlso}\relax
\code{\LinkA{.jnew}{.jnew}}, \code{\LinkA{.jsimplify}{.jsimplify}}
\end{SeeAlso}
\begin{Examples}
\begin{ExampleCode}
## Not run: 
z <- .jnew("java/lang/Boolean", TRUE)
.jfield(z, "TYPE")
## End(Not run)
\end{ExampleCode}
\end{Examples}

