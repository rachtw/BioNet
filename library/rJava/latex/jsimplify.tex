\HeaderA{jsimplify}{Converts Java object to a simple scalar if possible}{jsimplify}
\aliasA{.jsimplify}{jsimplify}{.jsimplify}
\keyword{interface}{jsimplify}
\begin{Description}\relax
\code{.jsimplify} attempts to convert Java objects that represent
simple scalars into corresponding scalar representation in R.
\end{Description}
\begin{Usage}
\begin{verbatim}
.jsimplify(o)
\end{verbatim}
\end{Usage}
\begin{Arguments}
\begin{ldescription}
\item[\code{o}] arbitrary object
\end{ldescription}
\end{Arguments}
\begin{Details}\relax
If \code{o} is not a Java object reference, \code{o} is returned
as-is. If \code{o} is a reference to a scalar object (such as single
integer, number, string or boolean) then the value of that object is
returned as R vector of the corresponding type and length one.

This function is used by \code{\LinkA{.jfield}{.jfield}} to simplify the results
of field access if required.

Currently there is no function inverse to this, the usual way to wrap
scalar values in Java references is to use \code{\LinkA{.jnew}{.jnew}} as the
corresponding constructor.
\end{Details}
\begin{Value}
Simple scalar or \code{o} unchanged.
\end{Value}
\begin{SeeAlso}\relax
\code{\LinkA{.jfield}{.jfield}}
\end{SeeAlso}
\begin{Examples}
\begin{ExampleCode}
## Not run: 
i <- .jnew("java/lang/Integer", as.integer(10))
print(i)
print(.jsimplify(i))
## End(Not run)
\end{ExampleCode}
\end{Examples}

