\HeaderA{jinit}{Initialize Java VM}{jinit}
\aliasA{.jinit}{jinit}{.jinit}
\keyword{interface}{jinit}
\begin{Description}\relax
\code{.jinit} initializes the Java Virtual Machine (JVM). This
function must be called before any rJava functions can be used.
\end{Description}
\begin{Usage}
\begin{verbatim}
.jinit(classpath = NULL, parameters = NULL, ..., silent = FALSE)
\end{verbatim}
\end{Usage}
\begin{Arguments}
\begin{ldescription}
\item[\code{classpath}] Any additional classes to include in the Java class
paths (i.e. locations of Java classes to use). This path will be
prepended to paths specified in the \code{CLASSPATH} environment
variable.
\item[\code{parameters}] character vector of parameters to be passed to
the virtual machine. They are implementation dependent and apply
to JDK version 1.2 or higher only. Please note that each parameter
must be in a separate element of the array, you cannot use a
space-separated string with multiple parameters.
\item[\code{...}] Other optional Java initialization parameters (implementation-dependent).
\item[\code{silent}] If set to \code{TRUE} no warnings are issued.
\end{ldescription}
\end{Arguments}
\begin{Details}\relax
Stating with version 0.3-8 rJava is now capable of modifying the class
path on the fly for certain Sun-based Java virtual machines, even when
attaching to an existing VM. However, this is done by exploiting the
way ClassLoader is implemented and may fail in the future. In general
it is officially not possible to change the class path of a running
VM.

At any rate, it is impossible to change any other VM parameters of a
running VM, so when using \code{.jinit} in a package, be generous with
limits and don't use VM parameters to unnecessarily restrict
resources.
\end{Details}
\begin{Value}
The return value is an integer specifying whether and how the VM was
initialized. Negative values indicate failure, zero denotes successful
initialization and positive values signify partially successful
initilization (i.e. the VM is up, but parameters or class path could
not be set due to an existing or incompatible VM).
\end{Value}
\begin{SeeAlso}\relax
\end{SeeAlso}
\begin{Examples}
\begin{ExampleCode}
## Not run: 
## set heap size limit to 512MB (see java -X) and
## use "myClasses.jar" as the class path
.jinit(classpath="myClasses.jar", parameters="-Xmx512m")
## End(Not run)
\end{ExampleCode}
\end{Examples}

